% !TEX root = ../main.tex

% Project report section

\section{Project report}
\vspace{1em}

\section*{Invariance testing via kernel hypothesis tests}
\vspace{1em}

\begin{abstract}
Work in the domain adaptation literature often reduce learning to generalize across domains to learning invariances to an assumed a priori set of transformations. We frame the problem of determining whether the data contradicts the assumed invariance to a transformation group in terms of a kernel hypothesis test. We formulate two settings of invariance testing and develop a tractable test statistic for each setting. We discuss the main considerations and challenges when designing kernel invariance tests.
\end{abstract}


\subsection{Introduction}

Recent work in the domain adaptation literature have framed the problem of learning to generalize across domains as a problem of learning conditional invariances to transformations acting on input data (e.g., \parencite{Gu:2019,Chen:2020:scm,Schwobel:2021}). Often, these works assume a priori a structure on the set of relevant transformations (e.g., in the form of a group or a set of groups) and propose a model that is trained from data to be (approximately) invariant to this structure. However, we question if this general procedure is missing a first step of checking whether the assumed invariance structure is even consistent with the data. If the assumed invariance contradicts what is observed, then the added invariance property of the model would be more harmful than beneficial for generalizing across domains. This is the same question that \textcite{Mouli:2021} addressed in their learning framework where a set of potential invariances are specified and the framework ``unlearns'' the invariances that contradict the data. In our work, rather than trying to learn (or unlearn) invariances from data, we focus on the problem of determining whether assuming the underlying data generating process being invariant would contradict the observed data.
\\

In this project, we examine the problem of determining potential conditional invariances (which we hereby refer to as \textit{invariance testing}) from a kernel hypothesis testing perspective. The objective and intuition behind the standard hypothesis testing framework appears well-matched to the objective of our problem as it answers the question: \textit{if we suppose the underlying generating distribution is invariant, is there evidence in the observed data that contradicts that assumption?} For practical invariance testing, we propose using kernel-based methods for their flexibility and ability to work with high-dimensional data. The main contributions of this project include
\begin{enumerate}

\item
a formulation of invariance testing in the case when only data from a single domain is available;

\item
a formulation of invariance testing in the case when data from two different domains are available;

\item
a practical procedure for estimating the test statistic and carrying out the hypothesis test under both formulations; and

\item
a discussion of the main considerations when designing kernel invariance tests.

\end{enumerate}

This project report is organized as follows: Section~\ref{sec:related} highlights related work in the literature and explains how our approach differs; Section~\ref{sec:background} introduces our used notation and provides additional background for groups and kernel hypothesis tests; Section~\ref{sec:condind} and Section~\ref{sec:twosample} present our formulations of invariance testing in the context of two distinct settings; and Section~\ref{sec:discussion} concludes the report with a discussion and reflection of the work in this project.


\subsection{Related work}\label{sec:related}

\textbf{Learning non-contradicting invariances from single-domain data.} This project is inspired by the work of \textcite{Mouli:2021} who proposed a method for learning counterfactual group-invariances in neural networks given only data from a single domain. Their method is based on specifying a set of potential invariances and using a regularized optimization objective to ``unlearn'' the invariances that contradict the data. In this work, we frame the problem of detecting contradictions in data to invariance of a specified group as a hypothesis testing problem.
\\

\textbf{Domain adaptation based on learning invariances.} The work by \textcite{Mouli:2021} falls more generally under the domain adaptation literature. This literature is broad and consist of works involving varying problem settings and approaches. Recent work that reduces domain adaptation to learning invariances include (but are not limited nor mutually exclusive to) those based on invariant neural networks \parencite{Li:2018,Gu:2019,Zhao:2019,Schwobel:2021}, invariant kernels \parencite{Li:2018:kernel,Ma:2019,Elesedy:2021:equivariant,Elesedy:2021}, and causal reasoning \parencite{Magliacane:2017,Chen:2020:scm}. The majority of these approaches involve learning invariances from available training data that come from multiple domains (sometimes artificially such as in data augmentation). Our work also shares the perspective of reducing domain adaptation to invariances. However, rather than learning invariances from data, our hypothesis testing approach determines whether a specified invariance is compatible with the data (which may come from only a single domain).
\\

\textbf{Kernel hypothesis testing.} Our invariance testing setup builds on the existing literature of kernel hypothesis testing, which include kernel methods for common one-sample problems \parencite{Zhang:2011,Doran:2014,Kellner:2015,Chwialkowski:2016,Jitkrittum:2020} and two-sample problems \parencite{Gretton:2007,Gretton:2012}. While the proposed approaches differ depending on the context and goal of their respective problems, most approaches are based on mapping empirical  distributions (and target distributions in the case of some one-sample problems) to a kernel mean embedding in some reproducing kernel Hilbert space and comparing embeddings through the \textit{maximum mean discrepancy} (MMD) test statistic \parencite{Harchaoui:2013}.
%The MMD test statistic is convenient for kernel-based hypothesis tests as its null distribution can be estimated using standard methods such as bootstrap \parencite{Gretton:2012}.
Our setup of the detecting invariance problem makes use of some of these existing kernel-based hypothesis tests.
\\

\textbf{Hypothesis testing for invariances.}
% An online search with the keywords \textit{testing for invariances} will mostly return literature related to measurement invariance, which generally refers to measurement constructs being interpreted in a conceptually similar matter across people groups (primarily in a psychometric context) \parencite{Vanderberg:2000}. While in this work we also test for invariances, we are concerned with outputs in a particular dataset being invariant to a mathematical group from a group theoretic sense.
Our proposed invariance tests are used to determine potential invariances to the actions of an algebraic group. From this perspective, our work is closely related to the literature on testing for symmetries in data \parencite{Henze:2003,NgatchouWandji:2009,Partlett:2015}. However, while the literature addresses the problem of detecting some suspected invariance in the data, our work addresses the different problem of determining whether assuming the underlying distribution being invariant would contradict the data.

\subsection{Background and notation} \label{sec:background}

We first provide some additional background that will be useful for understanding our proposed tests as well as the mathematical notation that we will use in this report. In both of our formulations, we use $\calX$ and $\calY$ to represent the (potentially high-dimensional) space of inputs and outputs, respectively. For a random variable $X$ with support $\calX$, we overload the notation $\P_X$ to represent both its distribution and the corresponding probability measure.
\\

\textbf{Transformation groups and orbits}. Let $\calG$ be a transformation group that measurably acts on the input space $\calX$. For $g\in\calG$, we denote the action of $g$ on $x\in\calX$ given by the measurable map $\Phi:\calG\times\calX\rightarrow\calX$ as $g\cdot x = \Phi(g,x)$. For a set $A\subset\calX$, we write $g\cdot A=\{g\cdot x: x\in A\}$. To refer to the distribution of $g^{-1}\cdot X$ for some random variable $X$ on $\calX$, we use the image measure denoted $\P_{gX}=\P_X\circ g^{-1}$, i.e., $\P_{gX}(dx)=\P_X(g^{-1}\cdot dx)$. We assume that $\calG$ is compact and therefore has an unique (normalized) Haar measure $\lambda$ \parencite[Theorem~2.10]{Folland:2016}. For $x\in\calX$, the \textit{orbit} of $x$ is defined as the set $x_\calG=\{g\cdot x:g\in\calG\}$. We denote the set of all orbits (which partitions $\calX$) as $\calX_\calG$.
\\

\textbf{Kernels and RKHS}. Let $\calH_\calX$ be a reproducing kernel Hilbert space (RKHS) of functions $f:\calX\rightarrow\bbR$ on which the evaluation functional $\delta_x:\calH_\calX\rightarrow\bbR$ is continuous for all $x\in\calX$, i.e., $\delta_x[f]=f(x)$ is bounded. Then by the Riesz representation theorem, there is a unique function $k_x\in\calH_\calX$ with the reproducing property, i.e., $f(x)=\langle f,k_x\rangle_{\calH_\calX}$ for all $f\in\calH_\calX$ where $\langle\cdot,\cdot\rangle_{\calH_\calX}$ is the inner product on $\calH_{\calX}$. The function $k_x=k(x,\cdot)$, where $k:\calX\times\calX\rightarrow\bbR$ is positive definite and symmetric with $k(x,x')=\langle k_x,k_{x'}\rangle_{\calH_\calX}$, is the reproducing kernel of $\calH_\calX$. For $x\in\calX$, the function $k_x=\varphi(x)$ can be viewed as a feature map $\varphi:\calX\rightarrow\bbR^d$ where $d$ may potentially be infinite. Therefore, evaluations of the reproducing kernel $k(x,x')=\langle \varphi(x),\varphi(x')\rangle$ may be viewed as inner products on the mapped feature space. For joint spaces $\calX\times\calY$, we consider the product kernel $k_{\calX\calY}((x,y),(x',y'))=k_\calX(x,x')k_\calY(y,y')$ that has feature map $\varphi_{\calX\calY}=\varphi_\calX\otimes\varphi_\calY$ where $\otimes$ denotes the tensor product. Note that we will drop the context space subscript in our notation when the context space is obvious, e.g., write $\calH$ instead of $\calH_\calX$, $k$ instead of $k_{\calX\times\calY}$, $\varphi$ instead of $\varphi_\calY$, etc.
\\

\textbf{Kernel embeddings}. For a distribution $\P_X\in\calH$, the \textit{kernel mean embedding} of $\P_X$ is defined as $\mu_X=\E[\varphi(X)]=\int_\calX\varphi(x)\P_X(dx)$. The kernel embedding $\mu_X$ is the unique element of $\calH$ such that $\E_X[f(X)]=\langle f,\mu_X\rangle_\calH$ for all $f\in\calH$. Let $\P_\calX$ denote the set of distributions on $\calX$. A kernel $k$ is said to be \textit{characteristic} if the kernel embedding $\mu_X:\P_\calX\rightarrow\calH$ is injective, i.e., each distribution corresponds to an unique point in the RKHS \parencite{Fukumizu:2007}. A kernel $k$ is translation-invariant if $k(x,x') = \Psi(x-x')$ for some bounded continuous positive definite function $\Psi:\bbR^d\rightarrow\bbR$. In this work, we assume that the kernels on the spaces $\calX$ and $\calY$ are both characteristic and translation-invariant, e.g., Gaussian kernels. Then under mild assumptions, the product kernel $k_{\calX\calY}=k_\calX k_\calY$ is also characteristic \parencite[Corollary 11]{Sriperumbudur:2010} and so the joint distribution kernel embedding $\mu_{XY}:\P_{\calX\calY}\rightarrow\calH$ is injective.
\\

\textbf{Kernel hypothesis tests and MMD}. Let $\calD^{(1)}=\{\xo_i\}_{i=1}^{n_1}$ and $\calD^{(2)}=\{\xt_i\}_{i=1}^{n_2}$ be two datasets. Two-sample kernel hypothesis tests are generally set up to test whether the distributions $\P_X^{(1)}$ and $\P_X^{(2)}$ that generated $\calD^{(1)}$ and $\calD^{(2)}$, respectively, are the same. These tests compare $\P_X^{(1)}$ and $\P_X^{(2)}$ by comparing their kernel embeddings through the \textit{maximum mean discrepancy} (MMD) test statistic \parencite{Gretton:2012}, which is given by
\[
\MMD(\P_X^{(1)},\P_X^{(2)}) = \left\|\mu_X^{(1)}-\mu_X^{(2)}\right\|_\calH \;.
\]
The MMD test statistic is particularly convenient in the two-sample case as its squared form can be empirically estimated through only kernel evaluations using the (biased) estimator
\begin{align*}
\hatMMD^2(\calD^{(1)},\calD^{(2)}) &= \left\|\frac{1}{n_1}\sum_{i=1}^{n_1}\varphi(\xo_i)-\frac{1}{n_2}\sum_{i=1}^{n_2}\varphi(\xt_i)\right\|_\calH^2 \\
&= \frac{1}{n_1^2}\onevec_{n_1}^T\bfK^{(1)}\onevec_{n_1} + \frac{1}{n_2^2}\onevec_{n_2}^T\bfK^{(2)}\onevec_{n_2} - \frac{2}{n_1n_2}\onevec_{n_1}^T\bfK^{(12)}\onevec_{n_2}
\end{align*}
where $(\bfK^{(1)})_{ij}=k(\xo_i,\xo_j)$, $(\bfK^{(2)})_{ij}=k(\xt_i,\xt_j)$ and $(\bfK^{(12)})_{ij}=k(\xo_i,\xt_j)$. Note that for joint distributions $\P_{XY}$ with feature map $\varphi(x_i)\otimes\varphi(y_i)$, using a product kernel decomposes the joint Gram matrices into element-wise products of marginal Gram matrices, i.e., $\bfK_{\calX\calY} = \bfK_\calX\odot\bfK_\calY$ where $\odot$ is the Hadamard product, $(\bfK_{\calX\calY}=k((x_i,y_i),(x_j,y_j))$, $(\bfK_\calX)_{ij}=k(x_i,x_j)$ and $(\bfK_\calY)_{ij}=k(y_i,y_j)$.
\\

Under the null where $\P_X^{(1)}=\P_X^{(2)}$, the distribution of $\hatMMD$ can be estimated through several ways \parencite{Gretton:2009}. For example, one method is based on using a Gamma approximation where the parameters of the distribution are defined in terms of moments of the biased estimator. Estimating the moments has a computational complexity of $O(n_1n_2)$, which is generally lower than that of other methods such as those based on Pearson curves (which has a cubic cost) \parencite{Gretton:2009}. Bootstrap is another option where the datasets are repeatedly merged, shuffled, and split, and the distribution is approximated by the empirical distribution of the estimated test statistics obtained over the random splits. Depending on how the distribution of $\hatMMD$ is estimated, the rejection region can be computed either parametrically or based on an empirical quantile. The general kernel hypothesis testing procedure is summarized in Algorithm~\ref{alg:test}.
\\

\begin{algorithm}[H]
\SetAlgoLined
\SetKwInOut{Input}{Input}
\SetKwInOut{Output}{Output}
\Input{datasets $\calD^{(1)}$, $\calD^{(2)}$; significance level $\alpha$}
\Output{reject (1) or no reject (0)}
\BlankLine
compute approximation $\hat{\P}_0$ of null distribution $\P_0$\;
compute critical value $q^*$ such that $\hat{\P}_0(\{q\in\bbR:q\geq q^*\})=1-\alpha$\;
compute estimate $\hatMMD(\calD^{(1)},\calD^{(2)})$\;
\If{$\hatMMD(\calD^{(1)},\calD^{(2)})>q^*$}{
return 1\;
}
return 0\;
\caption{Kernel hypothesis testing procedure}
\label{alg:test}
\end{algorithm}
\vspace{1em}

In this project, we present two formulations of invariance testing and discuss how to compute $\hatMMD$ (line~3 of Algorithm~\ref{alg:test}) in each. Once the test statistic has been estimated, the testing procedure outlined in Algorithm~\ref{alg:test} can then be applied.


\subsection{Invariance testing via conditional independence tests} \label{sec:condind}

In the setting of our first formulation of invariance testing, we assume that the transformation $g\in\calG$ acting on an input $x\in\calX$ is observable. We assume that different domains differ in the distribution $\P_G$ over $\calG$, but we have only collected data from a single domain. Our objective in this setting is to determine whether transforming the inputs influences the outputs $y\in\calY$. For example, suppose that we have a collection of images of digits 0--9 where most images are collected upright, but a few have been rotated by some factor of $90^o$ (a transformation of the image). Suppose that if we had collected images through some other method (a different domain), we may end up with a larger proportion of rotated images than upright images. The question we would like to answer is whether the label of the image depends on the orientation of the image. In this example, the label of the image should be dependent on the orientation as $180^o$ rotations of 6 should produce a 9, and vice versa. For invariance testing problems that fit under this setting, we could achieve our objective by reducing the problem to a conditional independence test.

\subsubsection{Setup and approach}

Suppose we are given a dataset $\calD=\{(x_i,g_i,y_i)\}_{i=1}^n$ collected from a single domain where $x_i\in\calX$ are inputs (in some canonical form), $y_i\in\calY$ are outputs, and $g_i\in\calG$ are observable transformations of the inputs for some known group $\calG$ acting on $\calX$. Assume that the joint distribution factorizes as $\P_{XYG}=\P_G\P_X\P_{Y|X,G}$. Assume that $\bbP_X$ and $\bbP_{Y|X,G}$ are fixed across environments but $\bbP_G$ may vary. Our goal is to determine whether the outputs $Y_i$ being conditionally invariant to the transformations $G_i$, i.e.,
\[
\P_{XYG}=\P_G\P_X\P_{Y|X,G} = \P_G\P_X\P_{Y|X}\;,
\]
would contradict what is observed in the given dataset.
\\

We set up this problem as a kernel conditional independence test where the null hypothesis says the outputs are conditionally invariant to transformation of the inputs and the alternative hypothesis says otherwise. We follow the permutation test approach described by \textcite{Doran:2014}. The intuition is that under the null, we should be able to permute the transformations $g_i$ in the dataset without disturbing the joint distribution. We can transform the conditional independence test into a two-sample test by randomly splitting the dataset $\calD$ in half to get subsets $\calD^{(1)}$ and $\calD^{(2)}$, permuting the transformations in $\calD^{(2)}$ (i.e., $\calD_\bfP^{(2)}=\{(x_i,g_{\bfP(i)},y_i)\}_{i=1}^{n_2}$ where $\bfP:\{1,\ldots,n_2\}\rightarrow\{1,\ldots,n_2\}$ is some permutation operator), and then comparing the empirical distributions of $\calD^{(1)}$ and $\calD_\bfP^{(2)}$ through the $\MMD$ given by
\[
\MMD(\P_{XYG}^{(1)},\P_{XY(\bfP G)}^{(2)}) = \left\|\mu_{XYG}^{(1)}-\mu_{XY(\bfP G)}^{(2)}\right\|_\calH \;.
\]
The squared $\MMD$ is estimated by
\begin{align*}
\hatMMD^2(\calD^{(1)},\calD_\bfP^{(2)}) &= \left\|\frac{1}{n_1}\sum_{i=1}^{n_1}\varphi(\xo_i,\yo_i,\go_i) - \frac{1}{n_2}\sum_{i=1}^{n_2}\varphi(\xt_i,\yt_i,\gtw_{\bfP(i)})\right\|_\calH^2 \\
&= \frac{1}{n_1^2}\onevec_{n_1}^T\bfK^{(1)}\onevec_{n_1} + \frac{1}{n_2^2}\onevec_{n_2}^T\bfK^{(2)}\onevec_{n_2} - \frac{2}{n_1n_2}\onevec_{n_1}^T\bfK^{(12)}\onevec_{n_2}
\end{align*}
where
\begin{align*}
(\bfK^{(1)})_{ij} &= k((\xo_i,\yo_i,\go_i),(\xo_j,\yo_j,\go_j)) \;, \\
(\bfK^{(2)})_{ij} &= k((\xt_i,\yt_i,\gtw_{\bfP(i)}),(\xt_j,\yt_j,\gtw_{\bfP(j)})) \;, \\
(\bfK^{(12)})_{ij} &= k((\xo_i,\yo_i,\go_i),(\xt_j,\yt_j,\gtw_{\bfP(j)})) \;.
\end{align*}
The procedure for computing $\hatMMD$ in the conditional independence test is summarized in Algorithm~\ref{alg:condind}.
\\

\begin{algorithm}[H]
\SetAlgoLined
\SetKwInOut{Input}{Input}
\SetKwInOut{Output}{Output}
\Input{dataset $\calD$}
\Output{estimate $\hatMMD$}
\BlankLine
randomly partition $\calD$ into subsets $\calD^{(1)}$ and $\calD^{(2)}$ of approximately equal size\;
generate permutation operator $\bfP$\;
apply permutation operator to $\calD^{(2)}$ to obtain $\calD_\bfP^{(2)}$\;
compute and return $\hatMMD(\calD^{(1)},\calD_\bfP^{(2)})$\;
\caption{Computing $\hatMMD$ in the kernel conditional independence test for invariance}
\label{alg:condind}
\end{algorithm}
\vspace{1em}

Note that \textcite{Doran:2014} work under a setup that involves a different conditional structure, and so their method requires learning a permutation $\bfP$ that minimizes the covariance between the two subsets. This step is unnecessary for the conditional structure in our setup and therefore we can generate the permutation randomly (with the caveat that the operator should not have $\bfP(i)=i$ for any $i$).
\\

Once $\hatMMD$ has been computed, the hypothesis testing procedure given in Algorithm~\ref{alg:test} can be applied. If the test rejects the null, then there is evidence in the dataset to suggest that transformations of the inputs are associated with the outputs, i.e., that the outputs are not conditionally invariant to the transformations.

\subsubsection{Discussion}

We conclude our first formulation with a few comments. Under our described setting, the transformations $g$ must be observable and the canonical forms of the inputs must be computable. Our proposed approach would not be feasible if these assumptions do not hold. We have also worked under the assumption that the dataset comes from a single domain. However, this approach generalizes to multi-domain problems as well. In this case of datasets from two domains, the two datasets can be merged to obtain the input dataset to Algorithm~\ref{alg:condind}.
\\

While we have framed the transformations $g$ as members of a group $\calG$, our setup and approach does not require $\calG$ to be an actual group. From this perspective, $G$ is simply another random variable with support $\calG$, and the problem reduces to a standard conditional independence problem. Possible modifications that can be made to Algorithm~\ref{alg:condind} include replacing the transformations in $\calD^{(2)}$ by new transformations randomly sampled from $\calG$, and transforming inputs by a particular $g'$ (i.e., replacing $\gtw$ by $g'\circ\gtw$). While these modifications make use of group properties, they do not seemingly offer much benefit over the permutation-based approach in most cases. Our next formulation will rely on $\calG$ on being a group in order to work.


\subsection{Invariance testing via two-sample tests} \label{sec:twosample}

In the setting of our second formulation, suppose that we have two datasets from two different domains. We assume that each domain is uniquely represented by a different action from some unknown group $\calG^*$, and we also assume that the outputs $y\in\calY$ are conditionally invariant to actions of this group on the inputs $x\in\calX$. Our objective in this setting is to determine whether a specified group $\calG$ is the unknown group $\calG^*$. For example, suppose that we have images for various objects. For each object, we have two images that are taken from two different angles (two domains). The two angles are roughly the same for every object. The question we would like to answer is whether the label of the image (i.e., the object in the image) depends on the angle from which the image is taken. If the label is truly invariant to the angle, then the label should be invariant to not only the two angles in our datasets but every possible angle (the group~$\calG^*$). We address these questions in problems with similar context using lifted two-sample tests.

\subsubsection{Setup and approach}

Suppose we are given datasets $\calD^{(j)}=\{(x_i^{(j)},y_i^{(j)})\}_{i=1}^{n_j}$ collected from two different domains $j\in\{1,2\}$ where $x_i\in\calX$ are inputs and $y_i\in\calY$ are outputs. Suppose that different domains correspond to different transformations $g$ from some unknown group $\calG^*$ acting on $\calX$. Assume that the joint distribution factorizes as $\P_{XY}=\P_X\P_{Y|X}$ where $\P_{Y|X}$ is $\calG^*$-invariant, i.e.,
\[
\bbP_{Y|X}(dy|x) = \bbP_{Y|X}(dy|g\cdot x)
\]
for all $g\in\calG^*$. Assume that $\P_{Y|X}$ is fixed across domains, and that $\P_X$ is fixed but equivariant across domains in the sense that $\P_X^{(1)}=\P_{g^*X}^{(2)}$ for some $g^*\in\calG^*$. Our goal is to determine whether a specified group $\calG$ (with a normalized Haar measure $\lambda$) is the unknown group $\calG^*$.
\\

We set up this problem as a kernel two-sample test for comparing distributions where the null hypothesis says that $\calG=\calG^*$ and the alternative hypothesis says otherwise. The idea is that under the null, the conditional distributions of the outputs should be invariant to actions $g\in\calG$ on the given inputs, and so the conditional distribution should be stationary on the $\calG$-orbits of the inputs. Then under our problem assumptions, if we compare the lifted joint distributions defined on $\calX_\calG$ (the $\calG$-orbits of $\calX$) rather than on $\calX$, we should see little discrepancy between the lifted distributions. We can obtain the distribution over $\calX_\calG$ by orbit-averaging the distributions \parencite{Eaton:2007}, i.e.,
\[
\P_{X_\calG Y} = \int_\calG\P_{(gX)Y}\lambda(dg) \;.
\]
The following result formalizes the equivalence of the orbit-averaged joint distributions under the null and problem assumptions.

\begin{proposition} \label{prop:equiv}
Under the distributional assumptions of the problem setting and the null assumption $\calG=\calG^*$, the orbit-averaged joint distributions $\P_{X_\calG Y}^{(1)}$ and $\P_{X_\calG Y}^{(2)}$ corresponding to the two datasets are equal almost everywhere.
\end{proposition}
\begin{proof}
Note that $\calX_\calG$ is a measurable space as $\calX$ is a measurable space and $\calG$ acts measurably on $\calX$. We then have
\begin{align*}
\P_{X_\calG Y}^{(1)} &= \int_{\calG}\P_{(gX)Y}^{(1)}\lambda(dg) \\
&= \int_{\calG}\P_{Y|gX}\P_{gX}^{(1)}\lambda(dg) \\
&= \int_{\calG}\P_{Y|gX}\P_{gg^*X}^{(1)}\lambda(dg) \tag{*} \\
&= \int_{\calG}\P_{Y|gX}\P_{gX}^{(2)}\lambda(dg) \\
&= \P_{X_\calG Y}^{(2)}
\end{align*}
where the line $(*)$ follows due to $\calG$-invariance of $\P_{Y|X}$ and the fact that $gg^*g^{-1}\in\calG$.
\end{proof}

Proposition~\ref{prop:equiv} says that under the null, the orbit-averaged joint distributions of the two datasets should look similar. Therefore, if the $\MMD$ test statistic comparing the orbit-averaged empirical distributions is estimated to have a large value, the null assumption does not hold and so we should reject the null. For this approach to work, we need estimate the kernel embeddings of the orbit-averaged joint distributions. Our next result establishes that the kernel embedding of the orbit-averaged joint distribution is equal to the orbit-averaged kernel embedding of the joint distribution.

\begin{proposition} \label{prop:embed}
The kernel embedding of an orbit-averaged joint distribution is equal to the orbit-averaged kernel embedding of the joint distribution, i.e.,
\[
\mu_{X_\calG Y} = \int_\calG \mu_{(gX)Y}\lambda(dg)\;.
\]
\end{proposition}
\begin{proof}
Consider the Haar-integration kernel \parencite{Haasdonk:2005} on $\calX\times\calY$ that is $\calG$-invariant in one tensor dimension given by
\[
\bar{k}((x,y),(x',y'))=\int_\calG\int_\calG k((g\cdot x,y), (g'\cdot x',y'))\lambda(dg)\lambda(dg')\;.
\]
Define $k_\calG$ be the kernel on $\calX_\calG\times\calY$ corresponding to $\bar{k}$, i.e.,
\[
k_\calG((x_\calG,y),\cdot) = \bar{k}((x,y),\cdot) \;.
\]
We then have
\begin{align*}
\mu_{X_\calG Y} &= \int_{\calX_\calG\times\calY}k_\calG((x_\calG, y),\cdot)\P_{X_\calG Y}(dx_\calG,dy) \\
&= \int_{\calX\times\calY}\bar{k}((x,y),\cdot)\int_\calG\bbP_{XY}(g^{-1}\cdot dx,dy)\lambda(dg) \\
&= \int_\calG\int_{\calX\times\calY}\bar{k}((g\cdot x, y),\cdot)\bbP_{XY}(dx,dy)\lambda(dg) \\
&= \int_{\calX\times\calY}\bar{k}((x,y),\cdot)\bbP_{XY}(dx,dy) \tag{*} \\
&= \int_\calG\int_{\calX\times\calY}k((g\cdot x, y),\cdot)\bbP_{XY}(dx,dy)\lambda(dg) \\
&= \int_\calG\mu_{(gX)Y}\lambda(dg)
\end{align*}
where the line $(*)$ follows because $\bar{k}$ is $\calG$-invariant (in one tensor dimension).
\end{proof}

The significance of Proposition~\ref{prop:embed} is that it says we can work directly with the non-averaged kernel embeddings and deal with the integration afterwards. The MMD test statistic based on comparing the orbit-averaged kernel embeddings is then
\begin{align*}
\MMD(\P_{X_\calG Y}^{(1)},\P_{X_\calG Y}^{(2)}) &= \left\|\int_\calG\mu_{(gX)Y}^{(1)}\lambda(dg) - \int_\calG\mu_{(gX)Y}^{(2)}\lambda(dg)\right\|_\calH \\
&= \left\|\int_\calG\mu_{(gX)Y}^{(1)} - \mu_{(gX)Y}^{(2)}\lambda(dg)\right\|_\calH \\
&= \left\|\int_\calG\E_{\P_{(gX)Y}^{(1)}}\left[\varphi(X^{(1)},Y^{(1)})\right] - \E_{\P_{(gX)Y}^{(2)}}\left[\varphi(X^{(2)},Y^{(2)})\right]\lambda(dg)\right\|_\calH \;.
\end{align*}
A (biased) estimator of the squared test statistic is then given by
\begin{align*}
\hatMMD^2(\calD^{(1)},\calD^{(2)}) &= \left\|\int_\calG\frac{1}{n_1}\sum_{i=1}^{n_1}\varphi(g\cdot \xo_i,\yo_i) - \frac{1}{n_2}\sum_{i=1}^{n_2}\varphi(g\cdot \xt_i,\yt_i)\lambda(dg)\right\|_\calH^2 \\
&= \int_\calG\int_\calG \frac{1}{n_1^2}\sum_{i=1}^{n_1}\sum_{j=1}^{n_1}k((g\cdot \xo_i,\yo_i),(g\cdot \xo_j,\yo_j)) \\
&\quad + \frac{1}{n_2^2}\sum_{i=1}^{n_2}\sum_{j=1}^{n_2}k((g'\cdot \xt_i, \yt_i),(g'\cdot \xt_j, \yt_j)) \\
&\quad - \frac{2}{n_1n_2}\sum_{i=1}^{n_1}\sum_{j=1}^{n_2}k((g\cdot \xo_i, \yo_i),(g'\cdot \xt_j, \yt_j)) \lambda(dg)\lambda(dg') \\
&= \int_\calG\int_\calG \frac{1}{{n_1}^2}\onevec_{n_1}^T\bfK^{(1)}\onevec_{n_1} + \frac{1}{n_2^2}\onevec_{n_2}^T\bfK^{(2)}\onevec_{n_2} - \frac{2}{n_1n_2}\onevec_{n_1}^T\bfK^{(12)}\onevec_{n_2} \lambda(dg)\lambda(dg')
\end{align*}
where the matrices are defined as
\begin{align*}
\bfK^{(1)}_{ij} &= k((g\cdot \xo_i,\yo_i),(g\cdot \xo_j,\yo_j)) \;, \\
\bfK^{(2)}_{ij} &= k((g'\cdot \xt_i, \yt_i),(g'\cdot \xt_j, \yt_j)) \;, \\
\bfK^{(12)}_{ij} &= k((g\cdot \xo_i, \yo_i),(g'\cdot \xt_j, \yt_j)) \;.
\end{align*}
Depending on $\calG$, we may or may not be able to compute $\hatMMD$ exactly. If $\calG$ is finite and small enough, then $\hatMMD$ may be computed by replacing the integrals with sums over $\calG$. If $\calG$ is too large or is continuous, then we can only estimate $\hatMMD$ by sampling from $\calG$. Let $G_m=\{(g_i,g_i')\}_{i=1}^m$ denote the collection of $2m$ transformations sampled from $\calG$. An estimator for $\hatMMD^2$ is given by
\begin{align*}
\dhatMMD^2(\calD^{(1)},\calD^{(2)},G_m) &= \frac{1}{m^2}\sum_{i=1}^m\sum_{j=1}^m \left(\frac{1}{{n_1}^2}\onevec_{n_1}^T\bfK_i^{(1)}\onevec_{n_1} + \frac{1}{n_2^2}\onevec_{n_2}^T\bfK_j^{(2)}\onevec_{n_2} - \frac{2}{n_1n_2}\onevec_{n_1}^T\bfK_{ij}^{(12)}\onevec_{n_2}\right) \\
&= \frac{1}{mn_1^2}\sum_{i=1}^m\onevec_{n_1}^T\bfK_i^{(1)}\onevec_{n_1} + \frac{1}{mn_2^2}\sum_{j=1}^m\onevec_{n_2}^T\bfK_j^{(2)}\onevec_{n_2} - \frac{2}{m^2n_1n_2}\sum_{i=1}^m\sum_{j=1}^m\onevec_{n_1}^T\bfK_{ij}^{(12)}\onevec_{n_2}
\end{align*}
where the matrices are now defined as
\begin{align*}
(\bfK_i^{(1)})_{k\ell} &= k((g_i\cdot \xo_k,\yo_k),(g_i\cdot \xo_\ell,\yo_\ell)) \;, \\
(\bfK_j^{(2)})_{k\ell} &= k((g_j'\cdot \xt_k, \yt_k),(g_j'\cdot \xt_\ell, \yt_\ell)) \;, \\
(\bfK_{ij}^{(12)})_{k\ell} &= k((g_i\cdot \xo_k, \yo_k),(g_j'\cdot \xt_\ell, \yt_\ell)) \;. 
\end{align*}

Note that the $\MMD$ in this setting should be viewed as comparing distributions on $\calX_\calG\times\calY$ rather than as comparing distributions on $\calX\times\calY$. From this perspective, the integral over $\calG$ does not change the null distribution of $\hatMMD$ and so the distribution can still be approximated using standard methods. Using $\dhatMMD$ to estimate $\hatMMD$ also does not change the null distribution, although it adds a second layer of approximation where now $\dhatMMD\rightarrow\hatMMD$ only as the number of $\calG$ samples $m\rightarrow\infty$. Algorithm~\ref{alg:twosamp} summarizes the procedure for estimating $\hatMMD$.
\\

\begin{algorithm}[H]
\SetAlgoLined
\SetKwInOut{Input}{Input}
\SetKwInOut{Output}{Output}
\Input{datasets $\calD^{(1)}$, $\calD^{(2)}$, group $\calG$}
\Output{estimate $\dhatMMD$}
\BlankLine
sample $2m$ transformations from $\calG$ and save them in collection $G_m$\;
initialize zero matrices $\bfK^{(1)}$, $\bfK^{(2)}$ and $\bfK^{(12)}$\;
\For{$i=\{1,\ldots,m\}$}{
compute $\bfK_i^{(1)}$ and $\bfK_i^{(2)}$\;
$\bfK^{(1)} \leftarrow \bfK^{(1)} + \bfK_i^{(1)}$\;
$\bfK^{(2)} \leftarrow \bfK^{(2)} + \bfK_i^{(2)}$\;
\For{$j=\{1,\ldots,m\}$}{
compute $\bfK_{ij}^{(12)}$\;
$\bfK^{(12)} \leftarrow \bfK^{(12)} + \bfK_{ij}^{(12)}$\;
}
}
compute and return $\dhatMMD(\calD^{(1)},\calD^{(2)}, G_m)=\sqrt{\frac{1}{mn_1^2}\onevec_{n_1}^T\bfK^{(1)}\onevec_{n_1} + \frac{1}{mn_2^2}\onevec_{n_2}^T\bfK^{(2)}\onevec_{n_2} - \frac{2}{m^2n_1n_2}\onevec_{n_1}^T\bfK^{(12)}\onevec_{n_2}}$\;
\caption{Estimating $\hatMMD$ in the kernel two-sample test for invariance}
\label{alg:twosamp}
\end{algorithm}
\vspace{1em}

\subsubsection{Discussion} \label{sec:twosamp:disc}

We conclude our second formulation with a brief discussion. The procedure outlined in Algorithm~\ref{alg:twosamp} is approximately $O(m^2n_1n_2)$, which is likely too expensive to be of practical use. We may be able to reduce the cost to $O(mn_1n_2)$ by sampling new transformations $g_i$ and $g_i'$ every iteration rather than pre-sampling two lists and iterating through the outer product. It is also notable that $\hatMMD$ and $\dhatMMD$ computationally simplify further if the kernel on $\calX$ satisfies $k(g\cdot x,x')=k(x,g^{-1}\cdot x')$. In this case, $\bfK^{(1)}$ and $\bfK^{(2)}$ are independent of $g$ and $g'$, respectively, and so $\hatMMD$ reduces to
\[
\hatMMD(\calD^{(1)},\calD^{(2)}) = \frac{1}{{n_1}^2}\onevec_{n_1}^T\bfK^{(1)}\onevec_{n_1} + \frac{1}{n_2^2}\onevec_{n_2}^T\bfK^{(2)}\onevec_{n_2} - \frac{2}{n_1n_2}\int_\calG\int_\calG\onevec_{n_1}^T\bfK^{(12)}\onevec_{n_2}\lambda(dg)\lambda(dg') \;,
\]
Likewise, $\dhatMMD$ reduces to
\[
\dhatMMD(\calD^{(1)},\calD^{(2)},G_m) = \frac{1}{n_1^2}\onevec_{n_1}^T\bfK^{(1)}\onevec_{n_1} + \frac{1}{n_2^2}\onevec_{n_2}^T\bfK^{(2)}\onevec_{n_2} - \frac{2}{m^2n_1n_2}\sum_{i=1}^m\sum_{j=1}^m\onevec_{n_1}^T\bfK_{ij}^{(12)}\onevec_{n_2} \;.
\]
Note that this simplification does not change the overall complexity of the procedure which comes from computing the cross term.
\\

It may be of interest to know whether a similar testing approach based on comparing a distribution to its orbit-averaged distribution would work, e.g., in the case when only a single dataset from one domain is available. Conceptually, it appears that such a test would be meaningful as existing results suggest that a function is invariant if and only if its projection onto the invariance space is itself \parencite[Lemma~1]{Elesedy:2021}. Practically, however, we cannot use the joint distribution for this comparison due to the effect of the marginal in the non-averaged distribution. A comparison based on conditional embeddings may work in theory, but conditional embeddings specify a family of distributions rather than a single distribution \parencite{Song:2013} and so it would be necessary to consider how that would be addressed in the kernel testing framework.

\iffalse
\begin{itemize}

\item
Notation: image of $\P_X$ under $g$: $\P_{gX} = \P_{X} \circ g^{-1}$.
\[
\int \P_{gX}(dx) f(x) = \int \P_X(dx) f(gx)
\]

\end{itemize}
\fi


\subsection{Reflection and conclusion} \label{sec:discussion}

We conclude this report with a discussion of our work and of possible directions for future work. We also reflect on our progress made throughout this project.
\\

In this project, we described two settings where it may be of interest to determine whether the underlying distribution that generated the data may be conditionally invariant to the actions of some group. In the first setting where only data from a single domain is available, we formulated the problem in terms of a kernel hypothesis test of conditional independence. In the second setting where data from two domains are available, we formulated the problem in terms of a lifted two-sample kernel hypothesis test. In both settings, we described how to estimate the MMD test statistic. Approximation of its null distribution and the remainder of the testing procedure then follows standard kernel hypothesis test methodology.
\\

The main challenge in this project was formulating a setting where both (1) the assumed conditional invariance is the property being tested and (2) the test statistic is tractable. As discussed in Section~\ref{sec:twosamp:disc}, because our objective is conditional invariance, it is natural to first consider kernel embeddings of conditional distributions. However, a conditional embedding only corresponds to a single distribution when given a fixed value, and so a different test statistic that compares distribution families rather than distributions would be necessary. We opted for comparing joint distributions to allow the use of the MMD, and our second formulation was only made possible after relaxing the single domain assumption that we started with following \textcite{Mouli:2021}.
\\

As our work in this project is relatively preliminary, there are many potentially fruitful directions of future work worth pursuing. We formulated two distinct settings of invariance testing. It would be worth considering how relaxing certain assumptions may change the approach, e.g., relaxing the marginal independence assumption between $X$ and $G$ in our first formulation, or relaxing the assumptions that we made on the kernel $k$ that we use throughout. With regards to the approach, projections onto invariant subspaces \parencite{Elesedy:2021:equivariant,Elesedy:2021,Mouli:2021} appear to be an idea worth exploring, and it would be interesting to see where they may have a role in our testing framework. For near-term work, it would be useful to empirically evaluate our formulations and proposed methods to verify that the theory works in practice, which is a component absent in this report due to the short time frame of this project.


\subsubsection*{Personal reflection}

The timeline for the project portion of this qualifying paper was approximately two weeks. I spent the majority of the first week reading recent kernel hypothesis testing literature and the related, and the majority of the second week formulating the two invariance testing settings. Given the constrained time frame, I believe that I have made good progress and have developed a reasonable foundation that can be built on after this project. While I was initially unfamiliar with kernel methods and the RKHS literature, I have since developed an appreciation for such methods and am looking forward to continuing this work.