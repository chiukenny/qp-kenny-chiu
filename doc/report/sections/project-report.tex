% !TEX root = ../main.tex

% Project report section

\section{Project report}

\subsection{Introduction}

\todo define rkhs

\todo
hypothesis test works well with checking for contradictions as tests rejects if there is evidence to contradict the null (invariance)

\subsection{Related work}

\textbf{Learning invariance to unknown groups from single-environment data.} This project is inspired by the work of \textcite{Mouli:2021} who proposed a method for learning counterfactual group-invariances in neural networks given only data from a single environment. Their method is based on having a specified set of potential groups to be invariant to, and using a regularized optimization objective to ``unlearn'' the groups that contradict the data. In this work, we frame the context of determining whether being invariant to a specified group contradicts the given dataset as a hypothesis testing problem.
\\

\textbf{Domain adaptation based on learning invariances.} The work by \textcite{Mouli:2021} falls more generally under the domain adaptation literature. This literature is broad and consist of works involving varying problem settings and approaches. Recent work that reduces domain adaptation to learning invariances include (but are not limited nor mutually exclusive to) those based on invariant neural networks \parencite{Li:2018,Zhao:2019,Schwobel:2021}, invariant kernels \parencite{Li:2018,Ma:2019,Elesedy:2021:equivariant,Elesedy:2021}, and causal reasoning \parencite{Magliacane:2017,Chen:2020:scm}. The majority of these approaches involve learning invariances from available training data that come from multiple domains. Our work also shares the perspective of reducing domain adaptation to invariances. However, rather than learning invariances from data, our hypothesis testing approach determines whether a specified invariance is compatible with the data (which may come from only a single domain).
\\

\textbf{Kernel hypothesis testing.} Our setup of testing for invariances builds on the existing literature of kernel hypothesis testing, which include kernel methods for common one-sample problems \parencite{Zhang:2011,Doran:2014,Kellner:2015,Chwialkowski:2016,Jitkrittum:2020} and two-sample problems \parencite{Gretton:2007,Gretton:2012}. While the proposed approaches differ depending on the context and goal of their respective problems, most approaches are based on mapping empirical  distributions (and non-empirical distributions in the case of some one-sample problems) to a kernel mean embedding in some RKHS and comparing embeddings through the \textit{maximum mean discrepancy} (MMD) test statistic \parencite{Harchaoui:2013}. The MMD test statistic is convenient for kernel-based hypothesis tests as its null distribution can be estimated using standard methods such as bootstrap \parencite{Gretton:2012}. Our setup of the detecting invariance problem makes use of some of these existing kernel-based hypothesis tests.
\\

\textbf{Hypothesis testing for invariances.}
% An online search with the keywords \textit{testing for invariances} will mostly return literature related to measurement invariance, which generally refers to measurement constructs being interpreted in a conceptually similar matter across people groups (primarily in a psychometric context) \parencite{Vanderberg:2000}. While in this work we also test for invariances, we are concerned with outputs in a particular dataset being invariant to a mathematical group from a group theoretic sense.
Our hypothesis testing setup is used to detect possible invariances in the data to actions of a mathematical group. From this perspective, our work is closely related to the literature on testing for symmetries in data \parencite{Henze:2003,NgatchouWandji:2009,Partlett:2015}. However, while the literature addresses the problem of determining whether there is some aspect of invariance in the data, our work addresses the different problem of determining whether assuming the underlying distribution being invariant would contradict the data.

\todo add bit about invariant hypothesis tests?

\subsection{Background and notation}

\subsubsection{Testing conditional invariances in data via kernel conditional independence tests}

\begin{itemize}

\item
Context: suppose we have a dataset $\calD=\{(x_i,g_i,y_i)\}_{i=1}^n$ collected from a single environment where $x_i\in\calX$ are inputs (in some canonical form), $y_i\in\calY$ are outputs, and $g_i\in\calG$ for some known group $\calG$ acting on $\calX$ are observable transformations on the inputs (e.g., image orientations, image backgrounds, etc.). Assume that the joint distribution factorizes as $\bbP_{XYG}=\bbP_G\bbP_X\bbP_{Y|X,G}$. Assume that $\bbP_X$ and $\bbP_{Y|X,G}$ are fixed across environments but $\bbP_G$ may vary.

\item
Objective: our goal is to determine whether the outputs $Y_i$ being conditionally invariant to the transformations $G_i$, i.e.,
\[
\bbP_{XYG}=\bbP_G\bbP_X\bbP_{Y|X,G} = \bbP_G\bbP_X\bbP_{Y|X}\;,
\]
would contradict what is observed in the given dataset.

\item
Proposed approach: we set up this problem as a kernel conditional independence test where the null hypothesis says the outputs are conditionally invariant and the alternative hypothesis says otherwise.
\\

We follow the permutation-based approach described in \parencite{Doran:2014}. The idea is that under the null, we should be able to permute the transformations $g_i$ in the dataset without disturbing the joint distribution. We can then transform the conditional independence test into a two-sample test by splitting the dataset in half, permuting the transformations in one half, and then comparing the empirical distributions of the two halves. It is expected that unlike in \parencite{Doran:2014}, our setup does not require learning the permutation matrix (although we may want to maximize the number of samples that have different transformations after permutation).
\\

For comparing the distributions, we can use the well-studied MMD test statistic for two-sample kernel tests \parencite{Gretton:2012}. Let $k_x$, $k_y$ and $k_g$ be characteristic, positive-definite kernel functions defined on their respective spaces, e.g., $k_x:\calX\times\calX\rightarrow\bbR$ and where $k_x$ has a corresponding feature map $\phi_x:\calX\rightarrow\calH_\calX$ such that $k_x(x,x')=\langle\phi_x(x),\phi_x(x')\rangle$ with $\calH_\calX$ being the RKHS of $k_x$. Define the product kernel $k_{xyg}((x,y,g),(x',y',g'))=k_x(x,x')k_y(y,y')k_g(g,g')$ over $\calX\times\calY\times\calG$ with feature map $\phi_x\otimes\phi_y\otimes\phi_g$ ($\otimes$ being the tensor product). Because the kernels are characteristic, the kernel mean embedding $\mu$ (estimated by $\hatmu$) maps distributions to unique functions in the RKHS of their respective spaces. Let $\bbP^{(1)}$ and $\bbP^{(2)}$ be the empirical distributions of the split dataset and let $\bfP$ be the permutation matrix. The MMD is then computed as
\[
\MMD(\bbP^{(1)},\bfP\bbP^{(2)}) = \left\|\hatmu(\bbP^{(1)})-\hatmu(\bfP\bbP^{(2)})\right\|_\calH^2\;.
\]
For small datasets, the null distribution of the MMD test statistic can be estimated via bootstrap where the original dataset is randomly split in half multiple times. If the null is not rejected, then the conditional invariance assumption does not significantly contradict the given data.

\item
Limitations: the transformations $g_i$ must be observable and canonical forms of the inputs must be computable.

\item
Other considerations: the setup does not seem to restrict the input and output spaces nor does it make assumptions on the transformations. The set of possible transformations may not necessarily need to be a group (reduces problem to standard conditional independence testing of three variables)?

\item
Multiple environments: if we had data from two environments $\calD^{(j)}=\{(x_i^{(j)},g_i^{(j)},y_i^{(j)})\}_{i=1}^n$ for $j\in\{1,2\}$, a two-sample permutation test is possible by just merging the datasets and randomly splitting in two.

\end{itemize}

\subsubsection{Testing conditional invariances in data via orbit-averaged kernel two-sample tests}

Given datasets from two environment, we can give an alternative formulation:

\begin{itemize}

\item
Context: suppose we have datasets $\calD^{(j)}=\{(x_i^{(j)},y_i^{(j)})\}_{i=1}^n$ collected from two different environments $j\in\{1,2\}$ where $x_i\in\calX$ are inputs and $y_i\in\calY$ are outputs. Suppose that different environments can be represented by different transformations $g$ from some unknown group $\calG^*$ acting on $\calX$. Assume that the joint distribution factorizes as $\bbP_{XY}=\bbP_X\bbP_{Y|X}$ where $\bbP_{Y|X}$ is $\calG^*$-invariant, i.e.,
\[
\bbP_{Y|X}(y|x) = \bbP_{Y|X}(y|g\cdot x)
\]
for all $g\in\calG^*$. Assume that $\bbP_{Y|X}$ is fixed across environments, and $\bbP_X$ is fixed but equivariant across environments in the sense that $\bbP_X^{(1)}(x)=\bbP_X^{(2)}(g\cdot x)$ where $\bbP_X^{(1)}$ and $\bbP_X^{(2)}$ correspond to the marginals of different environments. For convenience of notation, we write $\bbP_{g\cdot X}=\bbP_X(g\cdot X)$.

\item
Objective: our goal is to determine whether a specified $\calG$ with a Haar measure $\lambda$ is the ``correct'' group, i.e., $\calG=\calG^*$.

\iffalse
\todo If so, then the conditional distribution $\mathbb{P}_{Y|X}$ is invariant to, i.e., whether
\[
\bbP_{Y|X} = \bbP_{Y|g\cdot X}
\]
for all $g\in\calG$ and $X\in\calX$.
\fi

\item
Proposed approach: we set up this problem as a kernel hypothesis test for comparing distributions where the null hypothesis says that $\calG=\calG^*$ and the alternative hypothesis says otherwise.
\\

Let $\bbP_X^{(1)}$ and $\bbP_X^{(2)}$ denote the marginal distributions for the two environments where $\bbP_{g'\cdot X}^{(1)}=\bbP_X^{(2)}$ for some $g'\in\calG^*$. By assumptions from the setup and from working under the null, the orbit-averaged joint distributions should be equivalent, i.e.,
\begin{align*}
\int_{\calG}\bbP_{(g\cdot X)Y}^{(1)}\mathrm{d}\lambda(g) &=
\bbP_{Y|X}\int_{\calG}\bbP_{g\cdot X}^{(1)}\mathrm{d}\lambda(g) \\
&= \bbP_{Y|X}\int_{\calG}\bbP_{g\cdot g'\cdot X}^{(1)}\mathrm{d}\lambda(g) \\
&= \bbP_{Y|X}\int_{\calG}\bbP_{g\cdot X}^{(2)}\mathrm{d}\lambda(g) = \int_{\calG}\bbP_{(g\cdot X)Y}^{(2)}\mathrm{d}\lambda(g) \;.
\end{align*}

The MMD test statistic is based on comparing the orbit-averaged kernel mean embeddings (\todo how to estimate?). The null distribution should still be computable. (\todo).

\iffalse
From our paired setup, we have empirical distributions $\widehat{\bbP}_X^{(1)}=\widehat{\bbP}_{g'\cdot X}^{(2)}$. We can approximate the joint distributions by
\begin{align*}
\bbP_{XY}^{(1)} &= \bbP_X^{(1)}\bbP_{Y|X}^{(1)} \approx \widehat{\bbP}_X^{(1)}\bbP_{Y|X}^{(1)}\;, \\
\bbP_{(g'\cdot X)Y}^{(2)} &= \bbP_{g'\cdot X}^{(2)}\bbP_{Y|g'\cdot X}^{(2)} \approx \widehat{\bbP}_{g'\cdot X}^{(2)}\bbP_{Y|g'\cdot X}^{(2)}\;.
\end{align*}

Let $\calO$ be the orbit-averaging operator \parencite{Elesedy:2021}
\[
\calO f(x) = \int_\calG f(g\cdot x)\mathrm{d}\lambda(g)\;.
\]
\parencite{Elesedy:2021,Elesedy:2021:equivariant} show that a function $f$ is $\calG$-invariant if and only if $\calO f=f$. We can likely extend this argument to a function $g(x,y)$ with two inputs where $g$ is $\calG$-invariant in one input, i.e., rewriting $g(x,y)=g_y(x)$ and $g_y(x)$ being $\calG$-invariant if and only if $\calO g_y = g_y$. Applying this to distributions, we should have $\bbP_{Y|X}$ is invariant if and only if for all $y\in\calY$,
\[
\bbP_{Y|X}(y|x)=\int_\calG\bbP_{Y|X}(y|g\cdot x)\mathrm{d}\lambda(g)
\]
and so
\[
\bbP_{Y|X}(y|x)\bbP_X(x)=\int_\calG\mathbb{P}_{Y|X}(y|g\cdot x)\bbP_X(x)\mathrm{d}\lambda(g) \;.
\]
\fi

\item
Conceptual idea: the joint distribution $\bbP_{XY}$ can be reduced to $\bbP_{(\calG\cdot X)Y}$ where $\calG\cdot X$ are the $\calG$-orbits of $\calX$. The conditional distribution of $Y$ is invariant on these orbits.

\item
An approach based on one dataset and comparing the kernel mean embedding to its orbit-averaged mean embedding would not work because the marginal distributions would be different.

\item
\todo
\[
\mu_{(\calG\cdot X)Y} = \bbE_{(\calG\cdot X)Y}\left[\varphi(X)\otimes\varphi(Y)\right] = \int_{\calX\times\calY}\varphi(x)\otimes\varphi(y)\d\int_\calG\bbP_{(g\cdot X)Y}(x,y)\mathrm{d}\lambda(g)
\]

\end{itemize}

\iffalse
\todo

\begin{itemize}

\item
Context: suppose we have a dataset $\calD=\{(x_i,y_i)\}_{i=1}^n$ collected from a single environment where $x_i\in\calX$ are inputs and $y_i\in\calY$ are outputs. Assume that the joint distribution factorizes as $\bbP_{XY}=\bbP_X\bbP_{Y|X}$. Assume that $\bbP_{Y|X}$ is fixed across environments but $\bbP_X$ may vary by means of some unknown group $\calG$ acting on $\calX$, i.e., $\bbP_X\neq\bbP_{g\cdot X}$ for $g\in\calG$ (each element in an $X$-orbit corresponds to a different environment). Assume that $\calG$ is a compact group with a Haar measure $\lambda$.

\item
Objective: our goal is to determine whether the conditional distribution $\mathbb{P}_{Y|X}$ is invariant to transformations $g\in\calG$ that act on $X$, i.e.,
\[
\mathbb{P}_{Y|X}=\mathbb{P}_{Y|g\cdot X}\;.
\]

\item
Proposed approach: we set up this problem as a kernel hypothesis test for comparing distributions where the null hypothesis says that $\mathbb{P}_{Y|X}$ is $\calG$-invariant on $X$ and the alternative hypothesis says otherwise.
\\

Let $\calO$ be the orbit-averaging operator \parencite{Elesedy:2021}
\[
\calO f(x) = \int_\calG f(g\cdot x)\mathrm{d}\lambda(g)\;.
\]
\parencite{Elesedy:2021,Elesedy:2021:equivariant} show that a function $f$ is $\calG$-invariant if and only if $\calO f=f$. We can likely extend this argument to a function $g(x,y)$ with two inputs where $g$ is $\calG$-invariant in one input, i.e., rewriting $g(x,y)=g_y(x)$ and $g_y(x)$ being $\calG$-invariant if and only if $\calO g_y = g_y$. Applying this to distributions, we should have $\bbP_{Y|X}$ is invariant if and only if for all $y\in\calY$,
\[
\bbP_{Y|X}(y|x)=\int_\calG\bbP_{Y|X}(y|g\cdot x)\mathrm{d}\lambda(g)
\]
and so
\[
\bbP_{Y|X}(y|x)\bbP_X(x)=\int_\calG\mathbb{P}_{Y|X}(y|g\cdot x)\bbP_X(x)\mathrm{d}\lambda(g) \;.
\]

\end{itemize}
\fi


\subsection{Testing for invariances via conditional independence tests}

\subsection{Testing for invariances via two-sample tests}

\subsection{Discussion and reflection}