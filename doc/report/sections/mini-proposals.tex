% !TEX root = ../main.tex

% Mini-proposals section

\section{Mini-proposals}

% each mini-proposal gets its own subsection
\subsection{Proposal 1: Learning Counterfactual G-invariances from Single Environments with Continuous Linear Groups}

\begin{itemize}

\item
Finite groups $\rightarrow$ continuous linear groups, e.g., rotations, with assumptions to guarantee existence of Haar measure (e.g., locally compact?)

\item
(Elesedy 2021) looks at a RKHS and decomposes it into a space of G-invariant functions and a space of functions that vanish when averaged over G.

\item
Reynolds operator is a functional(?) of a group:
\[
\bar{T}_\calG = \frac{1}{|\calG|}\sum_{T\in\calG}T
\]
Orbit-averaging operator is a functional(?) of a group and a function (the representation?):
\[
\calO_{\calG,x}(f) = \int_\calG f(gx)d\lambda(g)
\]
How are these related? $\calO$ is an orthogonal projection on RKHS. Does the eigenspace of $\calO$ play a similar role to the eigenspace of Reynolds operator?

\item
Assuming orbit-averaging replaces Reynolds operator, question is how to compute the 1-eigenspaces in practice (Reynolds operator can be computed for finite group, but averaging via Haar measure?).

\item
Results analogous to Lemma 1, 2 (1-eigenspace corresponds to invariant vectors) and Theorem 3 (subspaces for sets of groups) would be needed.

\end{itemize}


% each mini-proposal gets its own subsection
\subsection{Proposal 2: Learning Counterfactual G-equivariances from Single Environments}

\begin{itemize}

\item
Context: same as \parencite{Mouli:2021}, but equivariance rather than invariance is desirable (e.g., orientation of image).

\item
Idea: 0-eigenspace corresponds to equivariant aspect of group. Transform input to group representation (basis?), feed into invariant representation, re-apply transformation to obtain equivariant output.

\item
Questions: how to handle multiple groups? Input and output spaces potentially different, need to learn input transformation and output transformation separately?

\item
The intersection of 0-eigenspace for group $i$ and 1-eigenspace for groups $j\neq i$ (and remove orthogonal projection onto overgroups?) correspond to space of vectors that are invariant to actions of groups $j$ but not group $i$. For a set of groups of size $k$, we end up with $k$ different subspaces? Or intersect with 1-eigenspace of all other groups to only get 1 subspace for set?

\item
Input transformation: just use projection on 1-eigenspace and feed into invariant representation.

\item
?Output transformation: want to use projection on 0-eigenspace (intersected with other 1-eigenspaces), but representation is on different space than input? Can this output transformation be learned using the same framework? Or just have two separate input network components that each take one of projection onto 0/1-eigenspace and sum the outputs/feed both into link function?

\end{itemize}

% ...


% TODO
