% !TEX root = ../main.tex

% Mini-proposals section

\section{Mini-proposals}

% each mini-proposal gets its own subsection
\subsection{Proposal 1: MY PROPOSAL TITLE} % enter your proposal title


% each mini-proposal gets its own subsection
\subsection{Proposal 2: MY OTHER PROPOSAL TITLE} % enter your proposal title

% ...


% TODO
\parencite{Mouli:2021} requires specifying known linear groups of transformations.
\begin{itemize}

\item
Any finite group is linear? What implications are there with infinite linear groups?

\item
If considering non-normal subgroups, then Theorem 2 does not hold. Have to work with CG-invariances rather than G-invariances? What exactly does this mean?

\item
Can something be down without assuming known groups? Consider largest possible linear overgroup and automatically learn the strongest invariances?

\item
Framework breaks down with non-groups. Can a non-group approach work? Can non-group structures be specified/linear? Can a variant of Theorem 2 be derived?

\item
Learning framework finds invariant subspace of group using properties (idempotency) of projection operator. When not working in linear spaces, is there an analagous property of non-linear operators?

\item
Does the chosen regularizing term make sense? Is it optimal? Does its differentiable approximation affect these properties?

\item
Can the framework be adapted for some equivariant objective? i.e., learning equivariant groups rather than invariant groups.

\item
The proposed algorithm is exponential as it involves the power set. Can it be reduced? Appendix D: ``...the algorithm stops after finding all the basis for the space vec$(\calX)$, it is unclear if the worst-case runtime occurs in practice.''

\end{itemize}