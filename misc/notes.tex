\documentclass[10pt]{article}
% header.tex
% this is where you load pacakges, specify custom formats, etc.

\usepackage[left=1in,right=1in,top=1in,footskip=25pt]{geometry} 
% \usepackage{changepage}
\usepackage{amsmath,amsthm,amssymb,amsfonts}
\usepackage{mathtools}
% enumitem for custom lists
\usepackage{enumitem}
% Load dsfont this to get proper indicator function (bold 1) with \mathds{1}:
\usepackage{dsfont}
\usepackage{centernot}

\usepackage[usenames,dvipsnames]{xcolor}

% set up commenting code (I will use this during marking)
\definecolor{CommentColor}{rgb}{0,.50,.50}
\newcounter{margincounter}
\newcommand{\displaycounter}{{\arabic{margincounter}}}
\newcommand{\incdisplaycounter}{{\stepcounter{margincounter}\arabic{margincounter}}}
\newcommand{\COMMENT}[1]{\textcolor{CommentColor}{$\,^{(\incdisplaycounter)}$}\marginpar{\scriptsize\textcolor{CommentColor}{ {\tiny $(\displaycounter)$} #1}}}

\usepackage{appendix}

% set up graphics
\usepackage{graphicx}
\DeclareGraphicsExtensions{.pdf,.png,.jpg}
\graphicspath{ {fig/} }

\usepackage[sorting=nyt,backend=biber,bibstyle=alphabetic,citestyle=alphabetic,giveninits=true]{biblatex}

\usepackage{fancyhdr}
\pagestyle{fancy}
\setlength{\headheight}{40pt}

%%%%%%%%%%%%%%%%%%%%%%%%%%%%%%%%%%%%%%%%%%%%%%%%%%%%%%%%%%%%%%%%%%%%%%%%%%%%%%%%%%%%
% most other packages you might use should be loaded before hyperref
%%%%%%%%%%%%%%%%%%%%%%%%%%%%%%%%%%%%%%%%%%%%%%%%%%%%%%%%%%%%%%%%%%%%%%%%%%%%%%%%%%%%

% Set up hyperlinks:
\definecolor{RefColor}{rgb}{0,0,.65}
\usepackage[colorlinks,linkcolor=RefColor,citecolor=RefColor,urlcolor=RefColor]{hyperref}

\usepackage[capitalize]{cleveref}
\crefname{appsec}{Appendix}{Appendices} % you can tell cleveref what to call things
% defs.tex
% this is where you define custom notation, commands, etc.


%%
% full alphabets of different styles
%%

% bf series
\def\bfA{\mathbf{A}}
\def\bfB{\mathbf{B}}
\def\bfC{\mathbf{C}}
\def\bfD{\mathbf{D}}
\def\bfE{\mathbf{E}}
\def\bfF{\mathbf{F}}
\def\bfG{\mathbf{G}}
\def\bfH{\mathbf{H}}
\def\bfI{\mathbf{I}}
\def\bfJ{\mathbf{J}}
\def\bfK{\mathbf{K}}
\def\bfL{\mathbf{L}}
\def\bfM{\mathbf{M}}
\def\bfN{\mathbf{N}}
\def\bfO{\mathbf{O}}
\def\bfP{\mathbf{P}}
\def\bfQ{\mathbf{Q}}
\def\bfR{\mathbf{R}}
\def\bfS{\mathbf{S}}
\def\bfT{\mathbf{T}}
\def\bfU{\mathbf{U}}
\def\bfV{\mathbf{V}}
\def\bfW{\mathbf{W}}
\def\bfX{\mathbf{X}}
\def\bfY{\mathbf{Y}}
\def\bfZ{\mathbf{Z}}

% bb series
\def\bbA{\mathbb{A}}
\def\bbB{\mathbb{B}}
\def\bbC{\mathbb{C}}
\def\bbD{\mathbb{D}}
\def\bbE{\mathbb{E}}
\def\bbF{\mathbb{F}}
\def\bbG{\mathbb{G}}
\def\bbH{\mathbb{H}}
\def\bbI{\mathbb{I}}
\def\bbJ{\mathbb{J}}
\def\bbK{\mathbb{K}}
\def\bbL{\mathbb{L}}
\def\bbM{\mathbb{M}}
\def\bbN{\mathbb{N}}
\def\bbO{\mathbb{O}}
\def\bbP{\mathbb{P}}
\def\bbQ{\mathbb{Q}}
\def\bbR{\mathbb{R}}
\def\bbS{\mathbb{S}}
\def\bbT{\mathbb{T}}
\def\bbU{\mathbb{U}}
\def\bbV{\mathbb{V}}
\def\bbW{\mathbb{W}}
\def\bbX{\mathbb{X}}
\def\bbY{\mathbb{Y}}
\def\bbZ{\mathbb{Z}}

% cal series
\def\calA{\mathcal{A}}
\def\calB{\mathcal{B}}
\def\calC{\mathcal{C}}
\def\calD{\mathcal{D}}
\def\calE{\mathcal{E}}
\def\calF{\mathcal{F}}
\def\calG{\mathcal{G}}
\def\calH{\mathcal{H}}
\def\calI{\mathcal{I}}
\def\calJ{\mathcal{J}}
\def\calK{\mathcal{K}}
\def\calL{\mathcal{L}}
\def\calM{\mathcal{M}}
\def\calN{\mathcal{N}}
\def\calO{\mathcal{O}}
\def\calP{\mathcal{P}}
\def\calQ{\mathcal{Q}}
\def\calR{\mathcal{R}}
\def\calS{\mathcal{S}}
\def\calT{\mathcal{T}}
\def\calU{\mathcal{U}}
\def\calV{\mathcal{V}}
\def\calW{\mathcal{W}}
\def\calX{\mathcal{X}}
\def\calY{\mathcal{Y}}
\def\calZ{\mathcal{Z}}


%%%%%%%%%%%%%%%%%%%%%%%%%%%%%%%%%%%%%%%%%%%%%%%%%%%%%%%%%%
% text short-cuts
\def\iid{i.i.d.\ } %i.i.d.
\def\ie{i.e.\ }
\def\eg{e.g.\ }
\def\Polya{P\'{o}lya\ }
%%%%%%%%%%%%%%%%%%%%%%%%%%%%%%%%%%%%%%%%%%%%%%%%%%%%%%%%%%

%%%%%%%%%%%%%%%%%%%%%%%%%%%%%%%%%%%%%%%%%%%%%%%%%%%%%%%%%%
% quasi-universal probabilistic and mathematical notation
% my preferences (modulo publication conventions, and clashes like random vectors):
%   vectors: bold, lowercase
%   matrices: bold, uppercase
%   operators: blackboard (e.g., \mathbb{E}), uppercase
%   sets, spaces: calligraphic, uppercase
%   random variables: normal font, uppercase
%   deterministic quantities: normal font, lowercase
%%%%%%%%%%%%%%%%%%%%%%%%%%%%%%%%%%%%%%%%%%%%%%%%%%%%%%%%%%

% operators
\def\P{\bbP} %fundamental probability
\def\E{\bbE} %expectation
% conditional expectation
\DeclarePairedDelimiterX\bigCond[2]{[}{]}{#1 \;\delimsize\vert\; #2}
\newcommand{\conditional}[3][]{\bbE_{#1}\bigCond*{#2}{#3}}
\def\Law{\mathcal{L}} %law; this is by convention in the literature
\def\indicator{\mathds{1}} % indicator function

% sets and groups
\def\borel{\calB} %Borel sets
\def\sigAlg{\calA} %sigma-algebra
\def\filtration{\calF} %filtration
\def\grp{\calG} %group

% binary relations
\def\condind{{\perp\!\!\!\perp}} %independence/conditional independence
\def\equdist{\stackrel{\text{\rm\tiny d}}{=}} %equal in distribution
\def\equas{\stackrel{\text{\rm\tiny a.s.}}{=}} %euqal amost surely
\def\simiid{\sim_{\mbox{\tiny iid}}} %sampled i.i.d

% common vectors and matrices
\def\onevec{\mathbf{1}}
\def\iden{\mathbf{I}} % identity matrix
\def\supp{\text{\rm supp}}

% misc
% floor and ceiling
\DeclarePairedDelimiter{\ceilpair}{\lceil}{\rceil}
\DeclarePairedDelimiter{\floor}{\lfloor}{\rfloor}
\newcommand{\argdot}{{\,\vcenter{\hbox{\tiny$\bullet$}}\,}} %generic argument dot
%%%%%%%%%%%%%%%%%%%%%%%%%%%%%%%%%%%%%%%%%%%%%%%%%%%%%%%%%%

%%%%%%%%%%%%%%%%%%%%%%%%%%%%%%%%%%%%%%%%%%%%%%%%%%%%%%%%%%
%% some distributions
% continuous
\def\UnifDist{\text{\rm Unif}}
\def\BetaDist{\text{\rm Beta}}
\def\ExpDist{\text{\rm Exp}}
\def\GammaDist{\text{\rm Gamma}}
% \def\GenGammaDist{\text{\rm GGa}} %Generalized Gamma

% discrete
\def\BernDist{\text{\rm Bernoulli}}
\def\BinomDist{\text{\rm Binomial}}
\def\PoissonPlus{\text{\rm Poisson}_{+}}
\def\PoissonDist{\text{\rm Poisson}}
\def\NBPlus{\text{\rm NB}_{+}}
\def\NBDist{\text{\rm NB}}
\def\GeomDist{\text{\rm Geom}}
% \def\CRP{\text{\rm CRP}}
% \def\EGP{\text{\rm EGP}}
% \def\MittagLeffler{\text{\rm ML}}
%%%%%%%%%%%%%%%%%%%%%%%%%%%%%%%%%%%%%%%%%%%%%%%%%%%%%%%%%%

%%%%%%%%%%%%%%%%%%%%%%%%%%%%%%%%%%%%%%%%%%%%%%%%%%%%%%%%%%
% Project-specific notation should go here
% (Because it's at the end of the file, it can overwrite anything that came before.)

%e.g.,
\def\Laplacian{\calL}
\def\P{\calP}

% combinatorial objects
\def\perm{\sigma} %fixed permutation
\def\Perm{\Sigma} %random permutation
\def\part{\pi} %fixed partition
\def\Part{\Pi} %random partition


%%%%%%%%%%%%%%%%%%%%%%%%%%%%%%%%%%%%%%%%%%%%%%%%%%%%%%%%%%

\begin{document}

\section{Proposal scratch notes}

\begin{itemize}

\item
$\calD$, $\calI$: (unknown) disjoint sets of indices that describe the groups of transformations that are relevant and irrelevant to the output, respectively. $\calD\cup\calI=\{1,\ldots,m\}$.

\item
$U_Y$, $U_\calI$, $U_\calD$, $\tildeU_\calI$: independent latent variables that influence the value of the variable(s) that they point to.

\item
$X^{\text{(hid)}}$: some unknown canonical form of the observed input $X$.  It is assumed that given $U_\calD$ and $U_\calI$, $X$ was obtained from an ordered sequence of transformations on the canonical form, i.e.,
\[
X=T_{U_\calD,U_\calI}\circ X^{\text{(hid)}}
\]
where transformations
\[
T_{U_\calD,U_\calI}=T_\calI^{(1)}\circ T_\calD^{(1)}\circ T_\calI^{(2)}\circ\ldots
\]
make up the overgroup $\calG_{\calD\cup\calI}$, $T_\calD^{(j)}$ is a transformation in group $\calG_j$ from the overgroup $\calG_\calD=\langle\cup_{j\in\calD}\calG_j\rangle$, and $T_\calI^{(i)}\in\calG_i\subset\calG_\calI=\langle\cup_{i\in\calI}\calG_i\rangle$. Note that $\calG_\calI$ is also assumed to be a normal subgroup of $\calG_{\calD\cup\calI}$.

\item
$Y$: observed output assumed to be generated by
\[
Y = h(X^{\text{(hid)}},U_\calD,U_Y)
\]
where $h$ is a deterministic function.

\item
$X_{U_\calI\leftarrow\tildeU_\calI}$: counterfactual variable to $X$ where $U_\calI$ has been replaced by $\tildeU_\calI$, i.e.,
\[
X_{U_\calI\leftarrow\tildeU_\calI} = T_{U_\calD,\tildeU_\calI}\circ X^{\text{(hid)}} \;.
\]

\item
Want CG-invariant representation
\[
\Gamma(X) = \Gamma(T_{U_\calD,U_\calI}\circ X^{\text{(hid)}}) = \Gamma(T_{U_\calD,\tildeU_\calI}\circ X^{\text{(hid)}}) =  \Gamma(X_{U_\calI\leftarrow\tildeU_\calI})
\]
When $\calG_\calI$ is a normal subgroup of $\calG_{\calD\cup\calI}$, G-invariant representation
\[
\Gamma(X) = \Gamma(T_\calI\circ X)
\]
for all $T_\calI\in\calG_\calI$ is sufficient.

\item
Because groups are finite linear automorphisms, each transformation $T$ is just a linear function and so Reynolds operator can be applied directly to the group actions
\[
\bar{T} = \frac{1}{|\calG|}\sum_{T\in\calG}T
\]
For continuous linear groups, orbit-average over a Haar measure $\lambda$
\[
\bar{T} = \int_\calG T\lambda(T)
\]
\todo: need to estimate the operator. Assume uniform Haar and sample?

\end{itemize}

\section{Kernel hypothesis test notes}

\begin{itemize}

\item
Kernel methods work on inner products of feature maps ($\phi:X\rightarrow\mathcal{H}$) of observations in the RKHS associated with kernel. Inner products may be computed without explicitly computing the high-dimensional feature map (``kernel trick'').

\item
Main challenge of designing kernel-based hypothesis tests is deriving large-sample distribution of test statistic under null.

\item
Gram matrix should be positive semidefinite. Satisfised if kernel is symmetric and positive semidefinite.

\item
Let $X$ be a r.v. with distribution $\mathbb{P}$. \href{https://en.wikipedia.org/wiki/Kernel_embedding_of_distributions}{Mean element} $\mu_\mathbb{P}$ associated with $X$ is unique element of RKHS $\mathcal{H}$ s.t. for all $f\in\mathcal{H}$,
\[
\langle\mu_\mathbb{P},f\rangle = \mathbb{E}_\mathbb{P}[f(X)]
\]
Covariance operator $\Sigma_\mathbb{P}:\mathcal{H}\times \mathcal{H}\rightarrow\mathbb{R}$ associated with $X$ is unique operator s.t. for all $f,g\in\mathcal{H}$,
\[
\langle f, \Sigma_\mathbb{P}g\rangle = \mathrm{Cov}(f(X),g(X)) = \mathbb{E}_\mathbb{P}[f(X),g(X)] - \langle\mu_\mathbb{P},f\rangle\langle\mu_\mathbb{P},g\rangle
\]
Empirical estimates of inner products that lead to estimates of element/operator are available.

\item
If $\mathrm{dim}(\mathcal{H}=\infty$, $\mu_\mathbb{P}$ has more significance than in classical statistics.

\item
Detecting (conditional?) invariances in single training environment via hypothesis testing. Explicitly testing for invariances? \href{https://www.sciencedirect.com/science/article/pii/S0047259X03000447}{Invariant testing}

\end{itemize}

\end{document}